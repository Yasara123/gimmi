\documentclass[a4paper]{article}
\usepackage[ngerman]{babel}
\usepackage[T1]{fontenc}

% Seitenformatierung
\usepackage{fancyhdr,setspace}

%Raender einstellen
\usepackage[left=2.5cm,right=3cm,top=2.5cm,bottom=2cm,includeheadfoot]{geometry}

%Kopfzeile usw.
\pagestyle{fancy}
\fancyhf{}

%Kopfzeile links bzw. innen
\fancyhead[L]{\nouppercase{\leftmark}}
%Kopfzeile rechts bzw. außen
\fancyhead[R]{\thepage}
%Linie oben
\renewcommand{\headrulewidth}{0.5pt}

\setlength{\parindent}{0pt}                  %Absatz-Beginn um 0pt einruecken
\setlength{\parskip}{5pt plus 2pt minus 1pt} %Abstand zwischen Absaetzen

% tabellen
\usepackage{tabularx}
\usepackage{booktabs}

%sollte für alle funktionieren
\usepackage[utf8]{inputenc}
\usepackage{hyperref}

%Bibliograghie
%\usepackage{natbib}
\bibliographystyle{ieeetr}

% Graphiken einbinden
\usepackage{graphicx}

%fußnoten mehrfach referenzierbar machen
\newcommand{\footnoteremember}[2]{\footnote{#2}\newcounter{#1}\setcounter{#1}{\value{footnote}}}
\newcommand{\footnoterecall}[1]{\footnotemark[\value{#1}]}
\begin{document}

% Wir binden das Uni-Logo auf der Titelseite ein
\begin{titlepage}
\begin{center}
\begin{figure}[htbp]
\begin{center}
\includegraphics[width=3.5cm]{img/UniLogo.png}
\end{center}
\end{figure}

% Name es Instituts
\vspace*{\fill}{
	Stiftung Universität Hildesheim \\ 
	Fachbereich III -- Sprach- und Informationswissenschaften
	\\Institut für Informationswissenschaft und Sprachtechnologie}

\vfill {{\Large Hausaufgabe 6 \\[5mm] Erstellen von Bi- und Trigrammen und
frequenzbasierte Untersuchungen am Beispiel von Zeitungstexten zum Thema
Ministerpräsident}}

{\normalsize \textbf {Sebastian Kastner, 202322, Email:
sebastian\_kastner@gmx.net}}

\vfill{Studiengang:\\ International Information Engineering}
\vfill {Dozenten: Gertrud Faaß, Ralph Koelle\\

(Institut für Informationswissenschaft und Sprachtechnologie)\\}
\end{center}
\end{titlepage}

\thispagestyle{empty}

% CQP:
% http://bulba.sdsu.edu/cqpman.ps
% http://www.ims.uni-stuttgart.de/projekte/CorpusWorkbench/CQPUserManual/HTML/
% http://www.ims.uni-stuttgart.de/projekte/CorpusWorkbench/CQPUserManual/HTML/node27.html#SECTION00830000000000000000
% http://www.ims.uni-stuttgart.de/projekte/CorpusWorkbench/CQPTutorial/cqp-tutorial.pdf

% STTS:
% http://www.ims.uni-stuttgart.de/projekte/corplex/TagSets/stts-table.html

% Adjektiv+Nomen ohne McAllister (Top 10) als LaTeX:
%	// Korpus öffnen
%	MPRAES;
%	// LaTeX Ausgabe Mode
%	set PrintMode latex;
%	// Query
%	@[pos='ADJ(A|D)']@[pos='N(E|N)' & word!='McAllister'];
%	// Filter, Ausgabe in Datei tabelle1.tex
%	group Last matchend word by match lemma > "tabelle1.tex";

% Häufigste Nomina ohne McAllister (Top 10) als LaTeX
%	// Korpus öffnen
%	MPRAES;
%	// LaTeX Ausgabe Mode
%	set PrintMode latex;
%	// Query
%	@[pos='N(E|N)' & word!='McAllister'];
%	// Filter, Ausgabe in Datei tabelle2.tex
%	group Last target lemma > "tabelle2.tex";

% Adjektive mit Suchwort (McAllister) (Top 10) als LaTeX
%	// Korpus öffnen
%	MPRAES;
%	// LaTeX Ausgabe Mode
%	set PrintMode latex;
%	// Query
%	@[pos='ADJ(A|D)']@[word='McAllister'];
%	// Filter, Ausgabe in Datei tabelle3.tex
%	group Last match lemma > "tabelle3.tex";

\section{CQP Ausgaben}

% tabelle 1: adjektive+nomen top10
\begin{table}[htpb]\label{t}
	\center
	\begin{tabularx}{0.6\textwidth}{llr}
		\toprule
		\textbf{Adjektiv} & \textbf{Nomen} & \textbf{Quantität}\\
		\midrule
		% tabelle muss auf nur-Zeilen gekürzt werden!
		% Adjektive+Nomen
% Die ersten 10 Zeilen der CQP Ausgabe
vergangen
 & Woche & 11 \\
öffentlich-rechtlich
 & Rundfunk & 7 \\
nordrhein-westfälisch
 & Ministerpräsident & 7 \\
norddeutsch
 & Landesbank & 6 \\
grün
 & Gentechnik & 5 \\
rein
 & Spekulation & 5 \\
liberaldemokratisch
 & Partei & 4 \\
klug
 & Entscheidungen & 4 \\
offen
 & Denkmals & 4 \\
früh
 & Außenminister & 4 \\
		\bottomrule
	\end{tabularx}
	\caption{Kollokation Adjektive + Nomen, die häufigsten zehn Kombinationen.}
	\label{tab:undefined}
\end{table}

% tabelle 2: nomina top10
\begin{table}[htpb]\label{t}
	\center
	\begin{tabularx}{0.35\textwidth}{lr}
		\toprule
		\textbf{Nomen} & \textbf{Quantität}\\
		\midrule
		% tabelle muss auf nur-Zeilen gekürzt werden!
		% Nomina
% Die ersten 10 Zeilen der CQP Ausgabe
Ministerpräsident
 & 252 \\
Wulff
 & 112 \\
CDU
 & 87 \\
Christian
 & 65 \\
Land
 & 62 \\
Jahr
 & 62 \\
Landesbank
 & 54 \\
SPD
 & 48 \\
Deutschland
 & 40 \\
Euro
 & 39 \\
		\bottomrule
	\end{tabularx}
	\caption{Nomina, die häufigsten zehn Vorkommen.}
	\label{tab:undefined}
\end{table}

% tabelle 3: nomina top10
\begin{table}[htpb]\label{t}
	\center
	\begin{tabularx}{0.35\textwidth}{lr}
		\toprule
		\textbf{Adjektiv} & \textbf{Quantität}\\
		\midrule
		% tabelle muss auf nur-Zeilen gekürzt werden!
		%\begin{tabular}{llr}
 & nordrhein-westfälisch
 & 7 \\
 & sächsisch
 & 7 \\
 & irakisch
 & 7 \\
 & bayerisch
 & 6 \\
 & damalig
 & 4 \\
 & hessisch
 & 2 \\
 & niederländisch
 & 2 \\
 & baden-württembergisch
 & 2 \\
 & griechisch
 & 2 \\
 & japanisch
 & 2 \\
 & italienisch
 & 2 \\
 & früh
 & 2 \\
 & niedersächsisch
 & 2 \\
 & amtierend
 & 2 \\
 & neu
 & 2 \\
%\end{tabular}

		\bottomrule
	\end{tabularx}
	\caption{Quantität der Adjektive, die in Kollokation mit dem Suchwort
	auftreten.}
	\label{tab:undefined}
\end{table}

\subsection{Subtitle}

Plain text.

\subsection{Another subtitle}

More plain text.

\end{document}

