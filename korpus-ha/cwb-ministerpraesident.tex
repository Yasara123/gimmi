\documentclass[%
	type=document,%
  	style=article,% book, article
  	media=print,
  	pages=oneside,%
  	prefixLecturer=Dozenten:,
  	author=multiple,
  	%draft,
]{unihildesheim}
\usepackage[utf8]{inputenc}
\usepackage[ngerman]{babel}
\graphicspath{{../img/}}

% META
\Contact{uni@jens-bertram.net}
\Copyright{cc:by-nc-sa}
\Date{08.07.2012}{08. Juli 2012}
\Lecturer{Gertrud Faaß PhD,\\Dr. Ralph Kölle}
\Seminar{Formalisierung\\[0.5em]{\small Fachbereich III -- Sprach- und
Informationswissenschaften \\Institut für Informationswissenschaft und
Sprachtechnologie}}
\Semester{Sommersemester 2012}
\Title{Hausaufgabe 6}
\Subtitle{Erstellen von Bi- und Trigrammen und
frequenzbasierte Untersuchungen am Beispiel von Zeitungstexten zum Thema
Ministerpräsident}
\Author{J. Bertram, S. Kastner}{Jens Bertram {\small(214508)}\\Sebastian Kastner
{\small(202322)}}
% /META

\begin{document}

\Titlepage
\cleardoublepage
\TOC
\cleardoublepage

%% Wir binden das Uni-Logo auf der Titelseite ein
\begin{titlepage}
\begin{center}
\begin{figure}[htbp]
\begin{center}
\includegraphics[width=3.5cm]{img/UniLogo.png}
\end{center}
\end{figure}

% Name es Instituts
\vspace*{\fill}{
	Stiftung Universität Hildesheim \\ 
	Fachbereich III -- Sprach- und Informationswissenschaften
	\\Institut für Informationswissenschaft und Sprachtechnologie}

\vfill {{\Large Hausaufgabe 6 \\[5mm] Erstellen von Bi- und Trigrammen und
frequenzbasierte Untersuchungen am Beispiel von Zeitungstexten zum Thema
Ministerpräsident}}

{\normalsize \textbf {Sebastian Kastner, 202322, Email:
sebastian\_kastner@gmx.net}}

\vfill{Studiengang:\\ International Information Engineering}
\vfill {Dozenten: Gertrud Faaß, Ralph Koelle\\

(Institut für Informationswissenschaft und Sprachtechnologie)\\}
\end{center}
\end{titlepage}

\thispagestyle{empty}

\section{Vorverarbeitung der Datei Ministerpräsident.txt}
An dieser Stelle sollen in einer Auflistung die Schritte beschrieben werden,
in denen die Ausgangsdatei, Ministerpräsident.txt, über Linux Tools auf der
Kommandozeile so umgeformt wurde, dass sie als CQP Korpus verarbeitet werden
konnte. Eine detaillierte Erklärung zu den verwendeten Befehlen ist in der
beiligenden Datei \textit{.info} zu finden.
\begin{enumerate}
  \item \textbf{Erzeugung von Unicode}
  \\Da die Ausgangsdatei in LATIN1 enkodiert ist
  wurde sie im ersten Schritt in UTF8 umgewandelt
  \item \textbf{Entfernen und Kommentieren von Metadaten}
   \\Die ersten 34 Zeilen, in denen sich nur Metadaten befinden und kein
  tatsächlicher Inhalt befinden wurden entfernt. Die Metadaten wurden zuvor
  manuell in die Datei \textit{.info} übernommen, damit sie später in CQP über
  den Befehle \verb info zur Verfügung stehen. Weiter wurden die Quellangaben
  \\ Außerdem wurden hier die Quellangaben, also auch Metadaten, auskommentiert,
  indem ein Hashtag (\#) vorangestellt wurde
  \item \textbf{Entfernung von Autorenangaben}
  \\ Zeilen mit Autorenangaben, also Zeilen, die mit "`Von \{GROßBUCHSTABE\}"`
  beginnen, wurden entfernt, da sie ebenfalls Metadaten darstellen und nicht zum
  eigentlichen Inhalt gehören
  \item \textbf{Tagging von Ortsangaben}
  \\ Im Ausgangstext wurden Ortsangaben häufig mit einem * eingeleitet. In
  diesem Schritt wurden wurden die Ortsangaben ausgelesen und anstelle eines
  führenden * mit einem p Tag <p ort="\{ORTSNAME\}"> ersetzt. Somit steht der
  Ortsname später als Meta Datum für den jeweiligen Absatz zur Verfügung.
  \item \textbf{Entfernung von COSMAS Match-Kennzeichnungen}
  \\ Hier wurden COSMAS Kennzeichnungen im Format <B>\{Kennzeichnung\}</> aus
  der Ausgangsdatei entfernt
  \item {Erstellung einer Liste der Quellen}
  \\ Die zuvor auskommentierten Quellangaben werden alle ausgelesen, sortiert
  und nummeriert. Das Ergebnis wird in einer separaten Datei gespeichert.
  \item \textbf{SKRIPT VON DER GERDL}
  \item \textbf{Hinzufügen des schließenden </p> Tags}
  \\ Hier werden zu den zuvor erzeugten <p> Tags mit Ortsangaben noch ein
  schließendes </p> Tag an das Ende des Absatzes hinzugefügt. Somit entstehen
  gekennzeichnete Absätze, die zusätzlich die Ortsangaben des jeweiligen
  Artikels liefern
  \item \textbf{Entfernung und Hinzufügen von src Tags}
  \\ Bedingt durch den vorigen Arbeitsschritt wurde ein schließendes </src> Tag
  am Anfang der Zieldatei erzeugt. Da kein öffnendes <src> Tag vorangeht ist
  dieses Tag überflüssig und wird dementsprechend aus der Datei gelöscht.
  Zusätzlich wird ein am Ende der Datei fehlendes schließendes </src> Tag
  eingefügt
  \item \textbf{Tokenisierung und Tagging}
  \\ Das somit erzeugte Format kann anschließend an einen Tokenisierer und einen
  Tree Tagger übergeben werden. Diese beiden Tools erzeugen Output, der von CQP
  verarbeitet werden kann
\end{enumerate}

\section{Analysen des erstellten Korpus}

Vor der Ausführung der hier ausgeführten Befehle wurde mit dem Kommando
\textit{set PrintMode latex;} das Ausgabeformat auf \LaTeX umgestellt, damit die
Ergebnisse ohne weitere Umformungen weiterverwendet werden konnten.


\subsection{Häufigste Adjektiv-Nomen Paare}
Für die Bestimmung der häufigsten Adjektiv-Nomen Paare, die nicht den
ursprünglichen Suchbegriff "`Ministerpräsident"' enthalten, wurden folgende
Befehle verwendet:
\begin{verbatim}
	> adj_n = @[pos="ADJ(A|D)"]@[pos="N(E|N)" & lemma!="Ministerpräsident.?"];
	> group adj_n matchend lemma by match lemma cut 4 > "tabelle1.tex";
\end{verbatim}

Mit dem ersten Befehl werden alle Adjektiv-Nomen Paare aus dem Korpus ausgelesen
und in der Variable adj\_n gespeichert. Im Folgenden soll der ausgeführte Befehl
kurz erläutert werden:
\begin{enumerate}
  \item \textit{@[pos="ADJ(A|D)"]} \\
			Das erste Wort muss mit dem POS Tag ADJA oder ADJD getaggt sein
  \item \textit{@[pos='N(E|N)' \& lemma!="Ministerpräsident.?"]} \\
			Das darauffolgende Wort muss mit dem POS Tag NE oder NE getaggt sein.
			Das Lemma des Nomens darf dabei nicht gleich Ministerpräsident sein. Da sich
			bei der Erstellung des Korpus bei dem Transport über verschiedene Systeme
			hinweg an das Ende der Lemmata mit \textit{\textasciicircum M} ein DOS
			Zeilenumbruch eingeschlichen. Mit \textit{.?} am Ende von Ministerpräsident wird das Lemma
			trotz dieses DOS Zeilenumbruchs gematched, aber Abwandlungen von
			Ministerpräsident, wie zum Beispiel Ministerpräsidentenkonferenz nicht.
			Der unerwünschte Zeilenumbruch bei den Lemmata wurde zu spät erkannt, als
			dass es noch möglich gewesen wäre ihn zu entfernen. Die Lemma Abfragen im
			weiteren wurden ebenfalls mit diesem Quickfix getätigt.
\end{enumerate}

Mit dem zweiten Befehl wurden die Ergebnisse des ersten Befehls nach der
Lemmakombination des ersten (also des Adjektivs) und des zweiten gematchten
Wortes (also das Nomen) gruppiert und nach der Häufigkeit der jeweiligen
Kombination absteigend angeordnet. Da nur die Ausgabe der 10 häufigsten
Kombinationen gefordert war, wurde die Anzahl der ausgegeben Kombinationen
mit \textit{cut} beschränkt. Durch \textit{cut 4} wurde die Ausgabe auf Kombinationen
beschränkt, die mindestens 4 mal auftreten. Der Wert 4 wurde zuvor manuell
bestimmt, indem der obige Befehl ohne den \textit{cut} Befehl verwendet wurde. Die
Ausgabe wurde in die Datei tabelle1.tex umgeleitet.

Das Ergebnis ist der Tabelle \ref{tab:adjektive_nomina} zu entnehmen.

% tabelle 1: adjektive+nomen top10
\begin{table}[htpb]\label{t}
	\center
	\begin{tabularx}{0.6\textwidth}{llr}
		\toprule
		\textbf{Adjektiv} & \textbf{Nomen} & \textbf{Quantität}\\
		\midrule
		% tabelle muss auf nur-Zeilen gekürzt werden!
		% Adjektive+Nomen
% Die ersten 10 Zeilen der CQP Ausgabe
vergangen
 & Woche & 11 \\
öffentlich-rechtlich
 & Rundfunk & 7 \\
nordrhein-westfälisch
 & Ministerpräsident & 7 \\
norddeutsch
 & Landesbank & 6 \\
grün
 & Gentechnik & 5 \\
rein
 & Spekulation & 5 \\
liberaldemokratisch
 & Partei & 4 \\
klug
 & Entscheidungen & 4 \\
offen
 & Denkmals & 4 \\
früh
 & Außenminister & 4 \\
		\bottomrule
	\end{tabularx}
	\caption{Kollokation Adjektive + Nomen, die häufigsten zehn Kombinationen.}
	\label{tab:adjektive_nomina}
\end{table}

\subsection{Häufigste Nomina}
Für die Bestimmung der häufigsten Nomina, die nicht gleich Ministerpräsident
sind, wurden folgende Befehle verwendet:
\begin{verbatim}
	> n = @[pos='N(E|N)' & lemma!='Ministerpräsident.?'];
	> group n target lemma cut 35 > "tabelle2.tex";
\end{verbatim}

Mit dem ersten Befehl werden alle Nomina (POS gleich NN oder NE), die nicht
gleich Ministerpräsident, beziehungsweise einer Deklination von
Ministerpräsident, sind abgefragt und in der Variable n gespeichert. Mit dem
zweiten Befehl wurden die so abgefragten Nomen nach ihrem Lemma grupiert und
nach ihrer Häufigkeit sortiert. Mit \textit{cut 35} wurde die Ausgabe auf Nomen
beschränkt, die mindestens 35 mal auftreten. Somit wurde die Ausgabe auf die
häufigsten 10 Nomen beschränkt, wie gefordert. Der Wert von 35 wurde wie im
vorigen Beispiel manuell durch einen unbeschränkten Aufruf des selben
Befehls ermittelt. Die Ausgabe wurde in die Datei tabelle2.tex umgeleitet. 

Das Ergebnis ist der Tabelle \ref{tab:nomina} zu entnehmen.
% tabelle 2: nomina top10
\begin{table}[htpb]\label{t}
	\center
	\begin{tabularx}{0.35\textwidth}{lr}
		\toprule
		\textbf{Nomen} & \textbf{Quantität}\\
		\midrule
		% tabelle muss auf nur-Zeilen gekürzt werden!
		% Nomina
% Die ersten 10 Zeilen der CQP Ausgabe
Ministerpräsident
 & 252 \\
Wulff
 & 112 \\
CDU
 & 87 \\
Christian
 & 65 \\
Land
 & 62 \\
Jahr
 & 62 \\
Landesbank
 & 54 \\
SPD
 & 48 \\
Deutschland
 & 40 \\
Euro
 & 39 \\
		\bottomrule
	\end{tabularx}
	\caption{Nomina, die häufigsten zehn Vorkommen.}
	\label{tab:nomina}
\end{table}

\subsection{Adjektiv in Kollokation mit "`Ministerpräsident"'}
Für die Bestimmung der häufigsten Nomina, die nicht gleich Ministerpräsident
sind, wurden folgende Befehle verwendet:
\begin{verbatim}
	> adj = @[pos='ADJ(A|D)']@[word='Ministerpräsident.*'];
	> group adj match lemma cut 2 > "tabelle3.tex";
\end{verbatim}
% tabelle 3: nomina top10
Mit dem ersten Befehl werden alle Adjektive (POS gleich ADJA oder ADJD)
ausgelesen, die von "`Ministerpräsident"' oder einer Deklination des Wortes
gefolgtw erden. Mit dem zweiten Befehl werden die Ergebnisse nach dem Lemma des
Adjektivs gruppiert. Mit \textit{cut 2} wird die Ausgabe auf Adjektive begrenzt, die
mindestens zweimal zusammen mit Ministerpräsident auftreten. Der Wert 2 wurde
wie in den vorigen Beispielen manuell ermittelt. Auch hier wurde die Ausgabe
wieder in eine Datei, diesmal tabelle3.tex, umgeleitet. 

Das Ergebnis ist der Tabelle \ref{tab:adj_mpraes} zu entnehmen.
\begin{table}[htpb]\label{t}
	\center
	\begin{tabularx}{0.5\textwidth}{lr}
		\toprule
		\textbf{Adjektiv} & \textbf{Quantität}\\
		\midrule
		% tabelle muss auf nur-Zeilen gekürzt werden!
		%\begin{tabular}{llr}
 & nordrhein-westfälisch
 & 7 \\
 & sächsisch
 & 7 \\
 & irakisch
 & 7 \\
 & bayerisch
 & 6 \\
 & damalig
 & 4 \\
 & hessisch
 & 2 \\
 & niederländisch
 & 2 \\
 & baden-württembergisch
 & 2 \\
 & griechisch
 & 2 \\
 & japanisch
 & 2 \\
 & italienisch
 & 2 \\
 & früh
 & 2 \\
 & niedersächsisch
 & 2 \\
 & amtierend
 & 2 \\
 & neu
 & 2 \\
%\end{tabular}

		\bottomrule
	\end{tabularx}
	\caption{Quantität der Adjektive, die in Kollokation mit dem Suchwort
	auftreten.}
	\label{tab:adj_mpraes}
\end{table}

\subsection{Subtitle}

Plain text.

\subsection{Another subtitle}

More plain text.

\end{document}

